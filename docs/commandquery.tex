\documentclass[letterpaper]{article}
\usepackage{tempora}%\usepackage{libertinus}%\usepackage{lmodern}
\usepackage{multicol}
\pagestyle{plain}
\usepackage{microtype}
\usepackage[breakable]{tcolorbox}
\usepackage{fullpage}
\usepackage{fullwidth}
\usepackage{color}
\usepackage{amsmath}
%\usepackage{titlesec}

\usepackage[hidelinks,colorlinks=true,urlcolor=blue]{hyperref}

%\titlespacing\subsection{0pt}{12pt plus 4pt minus 2pt}{2pt plus 2pt minus 2pt}

\DisableLigatures{encoding = *, family = *}
\pagenumbering{gobble}

\newenvironment{location}{\color{gray}}  %{\ignorespacesafterend}

\begin{document}
\begin{fullwidth}[width=\linewidth+2cm]%[width=\linewidth+2cm]
\begin{center}
\subsection*{Orc Library Command Query Code Flow Structure}
\vspace{3pt}
\end{center}
\noindent

  \subsection*{Command Query}

    A command query consists of many \textit{if-then} ($\beta - \varsigma$)
    pairs that operate in two distinct phases. Since all state checks happen in phase 1, phase-1 expressions cannot operate on phase-2 intermediate/modified state.

    \begin{equation*}
    \begin{aligned}
        & commandQuery \bigl( \\
        & \ \ \ \ \beta_1, \ ()\;\Rightarrow\;\{ \ \varsigma_1 \ \}, \\
        & \ \ \ \ \beta_2, \ ()\;\Rightarrow\;\{ \ \varsigma_2 \ \}, \\
        & \ \ \ \ ... \\
        & \bigr)
    \end{aligned}
    \end{equation*}

    \subsubsection*{1. Condition phase $\beta$}

    $\beta$ conditions are boolean expressions and all $\beta$ in a
    $commandQuery$ evaluate immediately.

    Their evaluation order is determined by the host language's function
    argument evaluation semantics.

    Each $\beta$ should be a pure boolean expression (no side effects).

    \subsubsection*{2. Command phase $\varsigma$}

    $\varsigma$ command code is wrapped in a parameterless lambda function
    to ensure it
    executes after all the command query's $\beta$ have finished evaluation and
    only if the associated $\beta$ evaluated to true.

\end{fullwidth}
\end{document}
