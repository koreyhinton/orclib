\documentclass[letterpaper]{article}
\usepackage{tempora}%\usepackage{libertinus}%\usepackage{lmodern}
\usepackage{multicol}
\pagestyle{plain}
\usepackage{microtype}
\usepackage[breakable]{tcolorbox}
\usepackage{fullpage}
\usepackage{fullwidth}
\usepackage{color}
\usepackage{amsmath}
%\usepackage{titlesec}

\usepackage[hidelinks,colorlinks=true,urlcolor=blue]{hyperref}

%\titlespacing\subsection{0pt}{12pt plus 4pt minus 2pt}{2pt plus 2pt minus 2pt}

\DisableLigatures{encoding = *, family = *}
\pagenumbering{gobble}

\newenvironment{location}{\color{gray}}  %{\ignorespacesafterend}

\begin{document}
\begin{fullwidth}[width=\linewidth+2cm]%[width=\linewidth+2cm]
\begin{center}
\subsection*{Orc Library Module Code Flow Structure}
\vspace{3pt}
\end{center}
\noindent

  \subsection*{Definition}

    \begin{equation*}
    \begin{aligned}
orc.module &= function(cb) \{
    var mod = \{loop:false\};
    cb(mod);
    while (mod.loop) \{ cb(mod) \};
    return mod;
\}
    \end{aligned}
    \end{equation*}

  \subsection*{Usage}

    \begin{equation*}
    \begin{aligned}
        moduleA = orc. & module\bigl( (m) \;\Rightarrow\; \{ \\
            & if(!m.loop) \{ m.loop=true; \} \\
            & if(m.loop) \{ m.i == null ? m.i = 0 : m.i++ \}
            & if(m.i == 5) \{ m.five = true; \}
        \}\bigr) \\
        moduleB = orc. & module\bigl( (m) \;\Rightarrow\; \{ \\
            & if (moduleA.five) \{ \}
        \}\bigr)
    \end{aligned}
    \end{equation*}


\end{fullwidth}
\end{document}
