\documentclass[letterpaper]{article}
\usepackage{tempora}%\usepackage{libertinus}%\usepackage{lmodern}
\usepackage{multicol}
\pagestyle{plain}
\usepackage{microtype}
\usepackage[breakable]{tcolorbox}
\usepackage{fullpage}
\usepackage{fullwidth}
\usepackage{color}
\usepackage{amsmath}
%\usepackage{titlesec}

\usepackage[hidelinks,colorlinks=true,urlcolor=blue]{hyperref}

%\titlespacing\subsection{0pt}{12pt plus 4pt minus 2pt}{2pt plus 2pt minus 2pt}

\DisableLigatures{encoding = *, family = *}
\pagenumbering{gobble}

\newenvironment{location}{\color{gray}}  %{\ignorespacesafterend}

\begin{document}
\begin{fullwidth}[width=\linewidth+2cm]%[width=\linewidth+2cm]
\begin{center}
\subsection*{Orc Library Command Query Code Flow Structure}
\vspace{3pt}
\end{center}
\noindent

  \subsection*{Frozen Expression}

    A frozen expression defines a boolean expression for a heavy compute that
    will only be evaluated once and will retain the first and only evaluated
    value on subsequent evaluations.

    \begin{equation*}
    \begin{aligned}
        & withFrozenExpression \bigl((e) \Rightarrow \{ \\
        & \ \ \ \ lazyExpVar = e(() \Rightarrow \beta) \\
        & \ \ \ \ commandQuery \bigl( \\
        & \ \ \ \ \ \ \ \ \beta_1 \ and \ lazyExpVar(), \ ()\;\Rightarrow\;\{ \ \varsigma_1 \ \}, \\
        & \ \ \ \ \ \ \ \ \beta_2 \ and \ lazyExpVar(), \ ()\;\Rightarrow\;\{ \ \varsigma_1 \ \}, \\
        & \ \ \ \ \ \ \ \ ... \\
        & \} \bigr)
    \end{aligned}
    \end{equation*}

\end{fullwidth}
\end{document}
